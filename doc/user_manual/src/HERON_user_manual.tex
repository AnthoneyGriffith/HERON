% \input{} or \include{} lines will added later
\documentclass[pdf,12pt]{../src/INLreport}
\usepackage{times}
\usepackage{longtable}
\usepackage[FIGBOTCAP,normal,bf,tight]{subfigure}
\usepackage{amsmath}
\usepackage{amssymb}
\usepackage[labelfont=bf]{caption}
\usepackage{pifont}
% \usepackage{enumerate}
\usepackage{listings}
\usepackage{fullpage}
\usepackage{xcolor}          % Using xcolor for more robust color specification
\usepackage{ifthen}          % For simple checking in newcommand blocks
\usepackage{textcomp}

% increase allowable depth of itemize lists
% - note that as of this writing we need a depth of 5.
\usepackage{enumitem}
\setlistdepth{7}
\setlist[itemize]{label=$\circ$}
\renewlist{itemize}{itemize}{7}
% end itemize depth adjustment

% suppress hbox overfull, underfull warnings
\hfuzz=\maxdimen
\hbadness=10000

%\usepackage{authblk}         % For making the author list look prettier
%\renewcommand\Authsep{,~\,}
% Custom colors
\definecolor{deepblue}{rgb}{0,0,0.5}
\definecolor{deepred}{rgb}{0.6,0,0}
\definecolor{deepgreen}{rgb}{0,0.5,0}
\definecolor{forestgreen}{RGB}{34,139,34}
\definecolor{orangered}{RGB}{239,134,64}
\definecolor{darkblue}{rgb}{0.0,0.0,0.6}
\definecolor{gray}{rgb}{0.4,0.4,0.4}
\lstset {
  basicstyle=\ttfamily,
  frame=single
}
\setcounter{secnumdepth}{5}
\lstdefinestyle{XML} {
    language=XML,
    extendedchars=true,
    breaklines=true,
    breakatwhitespace=true,
%    emph={name,dim,interactive,overwrite},
    emphstyle=\color{red},
    basicstyle=\ttfamily,
%    columns=fullflexible,
    commentstyle=\color{gray}\upshape,
    morestring=[b]",
    morecomment=[s]{<?}{?>},
    morecomment=[s][\color{forestgreen}]{<!--}{-->},
    keywordstyle=\color{cyan},
    stringstyle=\ttfamily\color{black},
    tagstyle=\color{darkblue}\bf\ttfamily,
    morekeywords={name,type},
%    morekeywords={name,attribute,source,variables,version,type,release,x,z,y,xlabel,ylabel,how,text,param1,param2,color,label},
}
\lstset{language=python,upquote=true}
\usepackage{titlesec}
\newcommand{\sectionbreak}{\clearpage}
\setcounter{secnumdepth}{4}
\usepackage[utf8]{inputenc}
% Default fixed font does not support bold face
\DeclareFixedFont{\ttb}{T1}{txtt}{bx}{n}{9} % for bold
\DeclareFixedFont{\ttm}{T1}{txtt}{m}{n}{9}  % for normal
\usepackage{listings}
\newcommand\pythonstyle{\lstset{
language=Python,
basicstyle=\ttm,
otherkeywords={self, none, return},             % Add keywords here
keywordstyle=\ttb\color{deepblue},
emph={MyClass,__init__},          % Custom highlighting
emphstyle=\ttb\color{deepred},    % Custom highlighting style
stringstyle=\color{deepgreen},
frame=tb,                         % Any extra options here
showstringspaces=false            %
}}
% Python environment
\lstnewenvironment{python}[1][]
{
\pythonstyle
\lstset{#1}
}
{}
% Python for external files
\newcommand\pythonexternal[2][]{{
\pythonstyle
\lstinputlisting[#1]{#2}}}
\lstnewenvironment{xml}
{}
{}
% Python for inline
\newcommand\pythoninline[1]{{\pythonstyle\lstinline!#1!}}
% Named Colors for the comments below (Attempted to match git symbol colors)
\definecolor{RScolor}{HTML}{8EB361}  % Sonat (adjusted for clarity)
\definecolor{DPMcolor}{HTML}{E28B8D} % Dan
\definecolor{JCcolor}{HTML}{82A8D9}  % Josh (adjusted for clarity)
\definecolor{AAcolor}{HTML}{8D7F44}  % Andrea
\definecolor{CRcolor}{HTML}{AC39CE}  % Cristian
\definecolor{RKcolor}{HTML}{3ECC8D}  % Bob (adjusted for clarity)
\definecolor{DMcolor}{HTML}{276605}  % Diego (adjusted for clarity)
\definecolor{PTcolor}{HTML}{990000}  % Paul
\def\DRAFT{} % Uncomment this if you want to see the notes people have been adding
% Comment command for developers (Should only be used under active development)
\ifdefined\DRAFT
  \newcommand{\nameLabeler}[3]{\textcolor{#2}{[[#1: #3]]}}
\else
  \newcommand{\nameLabeler}[3]{}
\fi
\newcommand{\alfoa}[1] {\nameLabeler{Andrea}{AAcolor}{#1}}
\newcommand{\cristr}[1] {\nameLabeler{Cristian}{CRcolor}{#1}}
\newcommand{\talbpaul}[1] {\nameLabeler{Paul}{PTcolor}{#1}}
% Commands for making the LaTeX a bit more uniform and cleaner
\newcommand{\TODO}[1]    {\textcolor{red}{\textit{(#1)}}}
\newcommand{\xmlAttrRequired}[1] {\textcolor{red}{\textbf{\texttt{#1}}}}
\newcommand{\xmlAttr}[1] {\textcolor{cyan}{\textbf{\texttt{#1}}}}
\newcommand{\xmlNodeRequired}[1] {\textcolor{deepblue}{\textbf{\texttt{<#1>}}}}
\newcommand{\xmlNode}[1] {\textcolor{darkblue}{\textbf{\texttt{<#1>}}}}
\newcommand{\xmlString}[1] {\textcolor{black}{\textbf{\texttt{'#1'}}}}
\newcommand{\xmlDesc}[1] {\textbf{\textit{#1}}} % Maybe a misnomer, but I am
                                                % using this to detail the data
                                                % type and necessity of an XML
                                                % node or attribute,
                                                % xmlDesc = XML description
\newcommand{\default}[1]{~\\*\textit{Default: #1}}
\newcommand{\nb} {\textcolor{deepgreen}{\textbf{~Note:}}~}
\usepackage{bm}
\newcommand{\tensor}[1]{{\bm{#1}}}
\renewcommand{\vec}{\bm}
\newcommand{\unit}[1]{\hat{\bm{#1}}}
\newcommand{\scalarunit}[1]{\hat{#1}}
\usepackage{booktabs}
\usepackage{stmaryrd}
\usepackage{hyperref}
\hypersetup{
    colorlinks,
    citecolor=black,
    filecolor=black,
    linkcolor=black,
    urlcolor=black
}
\newcommand{\wiki}{\href{https://github.com/idaholab/raven/wiki}{RAVEN wiki}}
\usepackage{cite}
\raggedbottom
\setcounter{secnumdepth}{5} % show 5 levels of subsection
\setcounter{tocdepth}{5} % include 5 levels of subsection in table of contents
\title{HERON User Manual}
\author{
\textbf{\textit{Project Manager:}}
 \\Cristian Rabiti\\
 \textbf{\textit{Principal Investigator and Technical Leader:}}
\\Paul W. Talbot\\
\textbf{\textit{Main Developers:}}
\\Paul W. Talbot\\
Abhinav Gairola\\
\textbf{\textit{Additional  Developers:}}
\\Jia Zhou
}
\date{}
\SANDnum{INL/EXT-20-58976, GDE-939}
\SANDprintDate{\today}
\SANDauthor{Cristian Rabiti, Paul W. Talbot, Abhinav Gairola, Jia Zhou}
\SANDreleaseType{Revision 1}
\def\component#1{\texttt{#1}}
% ---------------------------------------------------------------------------- %
\newcommand{\systemtau}{\tensor{\tau}_{\!\text{SUPG}}}
% Added by Sonat
\usepackage{placeins}
\usepackage{array}
\newcolumntype{L}[1]{>{\raggedright\let\newline\\\arraybackslash\hspace{0pt}}m{#1}}
\newcolumntype{C}[1]{>{\centering\let\newline\\\arraybackslash\hspace{0pt}}m{#1}}
\newcolumntype{R}[1]{>{\raggedleft\let\newline\\\arraybackslash\hspace{0pt}}m{#1}}
 \begin{document}
    \maketitle
    \SANDmain
    \tableofcontents
    \section{Introduction}
HERON is a generic software plugin for RAVEN to perform stochastic technoeconomic analysis of grid
energy-resource systems with economic drivers. The development targets analysis of electricity and
secondary product generation and consumption in regional balancing areas, including flexibility to
include arbitrary resources as well as arbitrary resource consumers and producers. HERON is
developed to drive optimization via economic drivers such as system cost minimization,
profitability, and net present value (NPV) maximization. As a plugin of RAVEN, HERON provides two
primary functions: the automatic generation of RAVEN workflows, and models for optimizing
high-resolution dispatch of arbitrary systems including resources, resource consumers, and resource
producers. HERON leverages the synthetic history training and generation tools, sampling workflows,
code Application Programming Interfaces (API), and optimization schemes.
    \section{ Installation and how to run}

\subsection{Installation}
As a plugin of RAVEN, HERON is installed as a submodule. RAVEN maintains up-to-date instructions for
plugin installation in its documentation
(\href{https://github.com/idaholab/raven/wiki/Plugins}{see the link to the Raven plugin
installation}.

\subsection{How to run}
Directly running HERON through RAVEN has not finished implementation. To run HERON, use Python to
run the script in \texttt{heron/src/main.py} with the HERON XML input as argument.

    \section{Input Structure}
In the following sections we describe the input structure in a general sense, with details in
following chapters.

\subsection{XML Input}
HERON makes use of the eXtensible Markup Language (XML) for its input structure, similar to RAVEN.
XML is made up of nodes, which have parameters and subnodes. For example:
\begin{lstlisting}[style=XML,morekeywords={class}]
  <node_tag par_name="par_value">
    <sub_tag sub_par_name="sub_par_value">sub_value</sub_tag>
  </node_tag>
\end{lstlisting}
The node's name (or tag) opens the XML element. In the example, we have two nodes, named
\xmlNode{node\_tag} and \xmlNode{sub\_tag}. The node \xmlNode{node\_tag} has a parameter with
name \xmlAttr{par\_name}. The parameter \xmlAttr{par\_name} has the value
\xmlString{par\_value}. Similarly, the \xmlNode{sub\_tag} node has a parameter and
corresponding value. The \xmlNode{sub\_tag} further has a value itself given by
\xmlString{sub\_value}.

We will use the terminology \xmlNode{node}, \xmlNode{subnode}, \xmlAttr{parameter}, and
\xmlString{value} to describe the input structure of HERON.

\subsection{Structure}
The HERON XML Input makes use of three main nodes within the root node \xmlNode{HERON}:
\begin{itemize}
  \item \xmlNode{Case}, in which general features of the desired solve are described, including general
    economics to apply, simulation properties such as project length and time stepping, and so forth.
  \item \xmlNode{Components}, in which the components of the grid system to analyze are defined,
    including their physical processes and economics.
  \item \xmlNode{DataGenerators}, in which data manipulation tools such as synthetic history
  generators and custom code functions.
  \item \xmlNode{TestInfo}, which is reserved for regression tests in HERON, describes the purpose
  of the test and information about the test's implementation.
\end{itemize}

Details for the three nodes used in HERON analyses are enumerated the following sections.
%INSERT_SECTIONS_HERE
    \providecommand*{\phantomsection}{}
    \phantomsection
    \addcontentsline{toc}{section}{References}
    \bibliographystyle{ieeetr}
    \bibliography{../../HERON_user_manual}
    \end{document}
