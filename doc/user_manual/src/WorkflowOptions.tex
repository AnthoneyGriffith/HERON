\section{Workflow Options}
HERON has the capability to run different workflows, which expand the flexibility and capabilities of this plugin. Currently, HERON input files control the workflow selection via the \xmlNode{workflow} node, which is a subnode of the \xmlNode{case} node. Each workflow approaches the Techno-economic Analysis (TEA) with a unique problem formulation and solution technique.

\subsection{Default}
The default HERON workflow primarily utilizes RAVEN. In this workflow, RAVEN runs RAVEN to solve a two-level representation of the TEA by utilizing both pyomo and RAVEN's gradient descent optimizer. To run this workflow, insert ``standard`` into the \xmlNode{workflow} node, or simply do not define this node.

\subsection{Monolithic Optimizer for Probabilistic Economic Dispatch (MOPED)}
This workflow formulates the problem as a single-level optimization problem. More specifically, MOPED utilizes TEAL and RAVEN's externalROMloader to generate and solve a pyomo object. To run this workflow, insert ``MOPED`` into the \xmlNode{workflow} node.
\noindent MOPED provides an alternate approach to solving the TEA provided by the input file. The solutions MOPED and the default workflow provide should be similar in standard cases.

\subsubsection{Motivations}
The primary motivations and potential benefits of MOPED include:
\begin{itemize}
    \item \textbf{Computational time:} In cases where the IES in question is following a cooperative dispatch heuristic (The standard dipatcher for the default workflow applies here), the single level formulation maintains the advantage of utilizing a more deterministic optimization algorithm
          (`ipopt`') than gradient search. This results from the gradient descent treating the NPV cost function as a black box with capacities as input and NPV as output. In constrast, MOPED's pyomo object has an algebraic expression generated with TEAL, allowing for more direct application of optimization techniques.
    \item \textbf{Seeding for more complicated scenarios:} In future versions of HERON, FARM will be available as a validation tool for HERON. FARM introduces new constraints that limit aspects of dispatch, such as ramping and setpoints, to ensure physical feasibility of the system's operation.
          For this use of HERON, MOPED could provide an initial solution input to FARM. This may reduce the number of iterations required to meet the validation criteria of the analysis.
    \item \textbf{Validation of default workflow/Confirmation of bilevel-monolithic equivalence:} Comparing the results between these two workflows provides a litmus test for the validity of either.
\end{itemize}

\subsubsection{Limitations}
MOPED is limited to the TEA's where the dispatch and capacity selection agents are cooperative. In other words, MOPED cannot solve analyses where maximizing dispatch value reduces the total NPV value. Possible scenarios include deregulated markets, direct competition, and agent-based dispatch.
 
Additionally, MOPED has limitations in terms of acceptable inputs, which currently include:
\begin{itemize}
    \item Custom functions for prices, VRE capacities, demand, etc.
    \item Components that do not start operation at project start
    \item Components with (component life x rebuild count $<$ project life)
    \item Components that produce, consume, or demand multiple resources
    \item Components with multiple cashflows of a same type
\end{itemize}
Development is focused on reducing the number of items on this list. The end goal of MOPED is to maintain the same capabilities as the default workflow.

\subsection{HERON Runs DISPATCHES (HERD)}
The Design Integration and Synthesis Platform to Advance Tightly Coupled Hybrid Energy Systems (DISPATCHES) repository can now be utilized using a HERON workflow (the github repository can be found here: \url{https://github.com/gmlc-dispatches/dispatches}). This HERON workflow builds a monolithic Pyomo object with cashflows generated through TEAL, similar to MOPED. In contrast, this workflow utilizes physics models from the Institute for the Design of Advanced Energy Systems (IDAES) which are linked using DISPATCHES flowsheets. 

DISPATCHES currently uses two workflows: a double-loop (a real-time and day-ahead market loop, not the RAVEN runs RAVEN loop) and multiperiod optimization.  HERD implements the multiperiod optimization workflow from DISPATCHES: it calls on an existing DISPATCHES flowsheet to create IDAES Pyomo physics models and runs a monolithic solve of multiple sampled market/weather scenarios. Available flowsheets pertain to three specific case studies in the DISPATCHES repository: a nuclear, renewables, and fossil fuel case. HERD implements the nuclear case with hydrogen production, storage, and combustion. 

\subsubsection{Installation}
DISPATCHES first needs to be installed within the same Python environment where RAVEN and the TEAL and HERON plugins are installed. The RAVEN Python environment must first be loaded
\begin{lstlisting}
  cd path/to/raven
  source ./scripts/establish_conda_env.sh --load
\end{lstlisting}
then DISPATCHES and all required packages should be installed within the RAVEN Python environment (assuming in this example that the DISPATCHES local repository is in the same project directory as RAVEN):
\begin{lstlisting}
  cd ../dispatches 
  python -m pip install -r requirements-dev.txt 
\end{lstlisting}
An additional flag, \texttt{--trusted-host=codeload.github.com}, can be added to the pip installation which adds github to trusted hosts to avoid a potential error. 

\subsubsection{How to Run}
To run this workflow, insert “DISPATCHES“ into the \xmlNode{workflow} node. The HERON XML Input script must include components that match the nuclear case:
\begin{itemize}
  \item a nuclear power plant,
  \item a proton exchange membrane electrolyzer for hydrogen production,
  \item a hydrogen storage tank,
  \item a hydrogen turbine for additional electricity production from hydrogen reserves,
  \item an electricity market, and
  \item a hydrogen pipeline.
\end{itemize}
A sample script can be found in
\begin{lstlisting}
  /tests/integration\_tests/workflows/HERD/nuclearCase\_ARMA.
\end{lstlisting} 

In the future, 