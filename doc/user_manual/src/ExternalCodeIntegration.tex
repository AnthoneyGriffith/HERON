\section{External Code Integration}
HERON can exchange data with other Integrated Energy Systems (IES) tools to conduct technical or economic analyses for various electricity market structures. More information about integrating HERON with other IES software is here: \url{https://ies.inl.gov/SitePages/FORCE.aspx}

Integrating HERON and other IES tools is still a work in progress. Currently, HERON can communicate with HYBRID only. HERON provides algorithms for analyzing the long-term viability of potential IES technologies, while HYBRID provides models to achieve high-resolution analysis over a short time. More information about HYBRID is here: \url{https://github.com/idaholab/HYBRID}


HYBRID models determine the fundamental properties of the IES components. These fundamental properties include transfer functions, operational ramping limits, and economics. The fundamental properties of generating units are used in HERON as constraints and economic drivers for long-term portfolio optimization. On the other side, the feasibility of the optimized energy portfolio, calculated by HERON, is analyzed using HYBRID to ensure that no physical or operational constraints are violated. HYBRID and HERON integration through automating the data exchange between HYBRID and HERON is demonstrated in the following subsections.

\subsection{Auto-loading the components' economic information from HYBRID to HERON}
HERON can load the economic information about the components of the grid. An example that demonstrates this capability can be found at: 
\begin{lstlisting}
/HERON/tests/integration_tests/mechanics/hybrid_load/
\end{lstlisting}

The Python script, that auto-loads the needed economic information from HYBRID to HERON, is named \path{hybrid2heron_economic.py}. This script can be found at:
\begin{lstlisting}
/HERON/src/Hybrid2Heron/
\end{lstlisting}


The \path{hybrid2heron_economic.py} script takes one command-line argument, which is the path of the initial HERON input XML file (or pre-input file) before loading any information from HYBRID

For example, the terminal command looks like this:
\begin{lstlisting}
python hybrid2heron_economic.py pre_heron_input.xml
\end{lstlisting}
A new HERON input XML file, \path{heron_input.xml}, is generated with all the subnodes under the \xmlNode{economics} node loaded from the HYBRID text files. More details about the initial HERON input XML file, the generated (loaded) input XML file and other files at the \path{/HERON/tests/integration_tests/mechanics/hybrid_load/} test are discussed in detail in the following subsections. 

\subsubsection{The initial HERON input XML file}
The initial HERON XML file, \path{pre_heron_input.xml}, structure should be similar to the typical HERON input XML file and must have
\begin{itemize}
\item A \xmlNode{Components} node: 
\item At least one \xmlNode{Component} sub-node under the \xmlNode{Components} node such as:
\begin{lstlisting}[style=XML,morekeywords={class}]
<Component name="component_name"> </Component>
\end{lstlisting}   

\item An empty \xmlNode{economics} node under the \xmlNode{Component} node with the path to the HYBRID text file that includes the needed information as follows:
\begin{lstlisting}[style=XML,morekeywords={class}]
<economics src="path/to/HYBRID/file"> </economics>
\end{lstlisting}   
\end{itemize}
If the \xmlNode{economics} node is not empty, it will remain unaltered when creating the final HERON XML file, \path{heron_input.xml}.

An example of the initial HERON XML file, \path{pre_heron_input.xml}, is located at \path{/HERON/tests/integration_tests/mechanics/hybrid_load/}. 
The \xmlNode{Components} node at the \path{pre_heron_input.xml} looks like this:
\begin{lstlisting}[style=XML,morekeywords={class}]
<Components>
  <Component name="source">
    <!--Other component subnodes-->
    <economics src="Costs/source/source.toml"></economics>
  </Component>
</Components>
\end{lstlisting}
In this example, the economic information of the component \xmlString{source} would be loaded from a text file whose path is \path{/Costs/source/source.toml}. Similarly, for any other component, the economic information can be provided from other text files. 

\subsubsection{The HYBRID text files}
The HYBRID text files are expected to have extensions such as .toml or .txt or .rtf and are located at: 
\path{/HERON/tests/integration_tests/mechanics/hybrid_load/Costs/sink/sink.toml} and 
\path{/HERON/tests/integration_tests/mechanics/hybrid_load/Costs/source/source.toml}. 

The HYBRID files do not have to be in the same folder with the \path{pre_heron_input.xml} as long as the value of the \xmlString{src} parameter at the \xmlNode{economics} node is the path to the corresponding HYBRID text file. The HYBRID text file structure looks like this:
\begin{lstlisting}
Lifetime = 30 #years
VOM = 0 # Just a placeholder to make sure it passes through
Activity = a
\end{lstlisting}

Each line, in the HYBRID text file, includes a variable name and its value plus a comment (if necessary). Any comments must start with the \verb|#| sign. The data should be appropriately auto-loaded from HYBRID text file to HERON XML file even if the text file includes additional irrelevant variables or additional comments at the top or the bottom of the file.

\subsubsection{The generated HERON Input XML file}
The generated input file, \path{heron_input.xml}, includes:
\begin{itemize}
\item All the information that was in the initial HERON input file,       \path{pre_heron_input.xml}
\item All the relevant variables from the HYBRID text files.
\item Default values for additional variables that are found neither at the \path{pre_heron_input.xml} nor at the HYBRID text files but are required by HERON to make sure that the input XML file is complete, and no required nodes or parameters are missing. The comments (warnings) inside the \path{heron_input.xml} tell the user if the values of specific variables are not provided by HYBRID, and if default values are assigned instead. The user should review these comments/warnings.
\end{itemize}

\subsubsection{HYBRID and HERON keywords}
The HYBRID keywords or the HYBRID variables that the \path{hybrid2heron_economic.py} code can identify to create the corresponding HERON nodes are listed in the CSV file, \path{HYBRID_HERON_keywords.csv}, which is located at: \path{/HERON/src/Hybrid2Heron}. Understanding this CSV file is essential, especially if the user plans to add more HYBRID variables or modify them. The CSV file, \path{HYBRID_HERON_keywords.csv} includes the following columns:


\begin{itemize}
    \item \textbf{HYBRID Keyword}: This column lists all the HYBRID variables' names that the Python script, \path{hybrid2heron_economic.py}, can identify. The variables' names in the HYBRID text files must be a subset of the HYBRID variables' names at the \path{HYBRID_HERON_keywords.csv}. Otherwise, the user can either change the variables' names in the \path{HYBRID_HERON_keywords.csv} file or in the HYBRID text files.    
    \item \textbf{Description}: The description or the definition of each HYBRID variable 
    \item \textbf{HERON (Node or Parameter)}: This column specifies if the HYBRID variable is corresponding to a HERON node or a node parameter in the HERON input XML file. \emph{N} refers to a node while \emph{P} refers to a parameter.
    \item \textbf{HERON Node}, \textbf{HERON Subnode} and \textbf{HERON Subsubnode}: These three columns specify the location of the HERON node corresponding to the HYBRID variable. For example, the HYBRID variable \verb|"Activity"| corresponds the sub-sub-node \xmlNode{activity} under the \xmlNode{driver} sub-node under the \xmlNode{CashFlow} node as follows:
    
    \begin{lstlisting}[style=XML,morekeywords={class}]
<CashFlow inflation="none" mult_target="FALSE" name="VOM" taxable="TRUE" type="repeating">
    <driver>
        <activity>a</activity>
    </driver>
</CashFlow>

    \end{lstlisting}
          Also, the HYBRID variable \verb|"VOM_inflation"| corresponds to a \emph{parameter(P)} or an attribute that is called \xmlString{inflation} at the node \xmlNode{CashFlow} under the \xmlNode{economics} node as illustrated in the \xmlNode{economics} node (above).
    
    \item \textbf{Belong to same node}: This column is intended to determine the list of sub-nodes or parameters that belong to the same node. We consider four primary nodes under the \xmlNode{economics} node, which are the component lifetime plus three types of cash flows: The Capital Expenditures (CAPEX) cash flow, the Fixed Operation and Maintenance (FOM) cash flow, and the Variable Operation and Maintenance (VOM) cash flow. These four primary nodes are listed under the \xmlNode{economics} node at the HERON input XML file are as follows:
    
\begin{minipage}{0.93\textwidth}
\begin{lstlisting}[style=XML,morekeywords={class}]
<economics>
  <lifetime>30</lifetime>
  <CashFlow inflation="none" mult_target="FALSE" name="capex" taxable="TRUE" type="one-time"></CashFlow>
  <CashFlow inflation="none" mult_target="FALSE" name="FOM" taxable="TRUE" type="repeating"></CashFlow>
  <CashFlow inflation="none" mult_target="FALSE" name="VOM" taxable="TRUE" type="repeating"></CashFlow>
</economics>
\end{lstlisting}
\end{minipage} \par

The numerical values under the \verb|"Belong to same node"| column are either \verb|0| or \verb|1| or \verb|2| or \verb|3| corresponding to the nodes \xmlNode{lifetime}, \xmlNode{CashFlow name="capex"}, \xmlNode{CashFlow name="FOM"}, \xmlNode{CashFlow name="VOM"} respectively. 

For example, since the HYBRID variable \verb|"capex_inflation"| corresponds to the parameter, \xmlString{inflation}, under the node \xmlNode{CashFlow name="capex"}, the corresponding numerical value under the \verb|"Belong to same node"| is \verb|"1"|. 

Similarly, the \verb|"Amortization_lifetime"| HYBRID variable corresponds to the HERON sub-node \xmlNode{depreciate} under \emph{all} the three cash flow nodes, the corresponding numerical value under the \verb|"Belong to same node"| is \verb|"1,2,3"|. 

Note that we consider three types only of cash flows, but the user can add additional cash flows to the \path{HYBRID_HERON_keywords.csv} file, if needed. 
     
 \item \textbf{Required if HYBRID keyword?}: This column determines if the HYBRID variable needs to be included when building the HERON input XML file even if this HYBRID variable is not provided by the HYBRID text files. 
 
 For example, the HYBRID variable, \verb|"Activity"|, will be included if the \verb|"VOM"| cash flow is present since the node \xmlNode{CashFlow name="VOM"} will be incomplete if the \xmlNode{activity} sub-node is missing. Therefore, for the HYBRID variable, \verb|"Activity"|, the corresponding value under the \verb|"Required if HYBRID keyword?"| column is \verb|"VOM"|
 
 \item \textbf{Default value for the required variable}: This column assigns a default value for any HYBRID variable whose value is not provided by the HYBRID text files if this HYBRID variable is required (see the column \verb|"Required if HYBRID keyword?"|). For example, the default value of the HYBRID variable, \verb|"Activity"|, if required, is \xmlString{electricity} 
\end{itemize}

\subsection{Auto-loading the optimized dispatches from HERON to HYBRID}
The optimized dispatches, that HERON calculates, can be converted to a text file compatible with HYBRID. An example that demonstrates this capability can be found at: 
\begin{lstlisting}
/HERON/tests/integration_tests/mechanics/optimizedDispatch2Hybrid/
\end{lstlisting}



This test demonstrates converting the HERON optimized components' variables (optimized dispatch) to a text file that is compatible with HYBRID via the following three steps:
\begin{enumerate}

\item \textbf{Producing the optimized dispatches}: Initially, the optimized dispatches are calculated via HERON through two steps (two optimization loops). An outer loop for optimizing the sizes of the grid components/units and an inner loop that calculates the optimal energy dispatch. These two steps (loops) create the optimized dispatches CSV file by running the usual HERON commands such as:
\begin{lstlisting}
~/projects/HERON/heron heron_input.xml
~/projects/raven/raven_framework outer.xml
\end{lstlisting}
Note that this HERON input XML file, \path{heron_input.xml}, is borrowed from \path{/HERON/tests/integration_tests/mechanics/debug_mode/} and is run in debug mode. The debug mode enables a reduced-size run (a single outer loop and a single inner loop). 

The output of the previous two commands is the optimized dispatch printed in
\begin{lstlisting}
optimizedDispatch2Hybrid/Debug_Run_o/dispatch_print.csv
\end{lstlisting}



\item \textbf{Creating the user-input file for the HYBRID user}: This step enables the HYBRID user to change the names, if necessary, of the optimized components' variables and the capacities of the components. The HYBRID user input file is created by running the code \path{create_user_input.py} which extracts the optimized dispatch outputs from the dispatches CSV file and the list of capacities of the components from the HERON input XML file.
 
 
The \path{create_user_input.py} takes two arguments: The HERON input XML file and the optimized dispatch outputs CSV file. For example, use the following command:
 \begin{lstlisting}
python create_user_input.py heron_input.xml Debug_Run_o/
dispatch_print.csv
  \end{lstlisting}
  
 
 
This step creates \path{user_input.txt} that the HYBRID user should modify or review before moving to the next step. The \path{user_input.txt} includes a list of the HERON variables and their corresponding HYBRID variables (that the HYBRID user may change) plus the location of \path{all_variables_file}. The \path{all_variables_file} is a file that includes all the HYBRID variables. All the capacities and dispatches should be a subset of this file's variables. This file helps to ensure that the HYBRID capacities and dispatches are identified by HYBRID. Currently the \path{all_variables_file} is {"dsfinal.txt"} which is located in 
\begin{lstlisting}
/HERON/tests/integration_tests/mechanics/
optimizedDispatch2Hybrid/dsfinal.txt
\end{lstlisting}
The user can change the file location in the \path{user_input.txt}.    
  


\item \textbf{Creating an optimized dispatches file for HYBRID}: A text file containing the optimized dispatches and capacities of the components is created to be compatible with HYBRID using the code \path{export2Hybrid.py}. The variables' names in the generated (auto-loaded) HYBRID text file are borrowed from the \path{user_input.txt} and the \path{user_input.txt} must be in the same folder as the \path{export2Hybrid.py}. 
 The \path{export2Hybrid.py} extracts the values of the capacities of the components from the HERON input file, \path{heron_input.xml}, and extracts the most challenging scenario from the optimized dispatch CSV file, \path{Debug_Run_o/dispatch_print.csv}. 
 
 We assume that the most challenging scenario is the scenario (of a specific year and sample) with the highest rate of change of any components' resource (variable). This most challenging scenario is selected as follows:
   

\begin{itemize}
 \item For each scenario, calculate the rate of change at each time step for each component variable
  \item For each scenario, calculate the maximum rate of change over all the time steps over all the variables. 
   \item Select the scenario with the highest maximum rate of change to be the maximum scenario.
 \end{itemize}
 


Note that we assume that all the components' variables in the \path{dispatch_print.csv} have the same units. 
 
 The \path{export2Hybrid.py} is run with two arguments: The HERON input XML file and the optimized dispatch outputs CSV file. For example, use the following command:
 
 \begin{lstlisting}
python export2Hybrid.py heron_input.xml Debug_Run_o/
dispatch_print.csv
 \end{lstlisting}

The output will be a text file compatible with HYBRID:    \path{hybrid_compatible_dispatch.txt}. The terminal will display warning messages if the HYBRID dispatches/capacities are not a subset of the variables list in the \path{user_input.txt}.    

\end{enumerate}
