\section{Introduction}
\subsection{System Purpose}

The \textbf{\textit{HERON}} plug-in is a generic tool for techno-economic analysis of resource distribution system in Requlated and Deregulated Market.
\\RAVEN is a flexible and multi-purpose uncertainty quantification (UQ), regression analysis, probabilistic risk assessment 
(PRA), data analysis and model optimization software.  Depending on the tasks to be accomplished and on the 
probabilistic
 characterization of the problem, RAVEN perturbs (Monte-Carlo, Latin hyper-cube, reliability surface search, etc.) the
 response of the system under consideration by altering its own parameters. 
 The data generated by the sampling process is analyzed using classical statistical
 and more advanced data mining approaches. RAVEN also manages the parallel dispatching (i.e. both on
 desktop/workstation and large High-Performance Computing machines) of the software representing the physical 
 model.
 For more information about the RAVEN software, see ~\cite{RAVENuserManual} and the RAVEN website (\url{raven.inl.gov})
\\The  \textbf{\textit{HERON}} is one of the first developed RAVEN plug-ins. 
A RAVEN plug-in is a software/module/library that has been developed to be linked to RAVEN at run-time, using the RAVEN APIs.


\subsection{System Scope}

The \textbf{\textit{HERON}} is a project to construct and run RAVEN workflows solving complex resource allocation problems 
to meet target economic goals. It levearges probablistic analysis tool RAVEN to run contructed workflows.
It requires Cashflow plug-in of RAVEN to enable economic analysis.

 The main objective of the module (in conjunction with the RAVEN software) is to assist the engineer/user to perform
component sizing optimization using stochastic optimization of resource allocation. Components are entities that produce
or consume resource that are tracked in analysis.

In other words, the  \textbf{\textit{HERON}} plug-in (driven by RAVEN) is aimed to be employed for:
\begin{itemize}
  \item Automated RAVEN workflow generation for grid energy system techno-economic analysis
  \item Grid system dispatch optimization on stochastic signals
  \item Regulated and deregulated market analysis
  \item Component sizing optimization
\end{itemize}


\subsection{User Characteristics}

The users of the \textbf{\textit{HERON}} plug-in are expected to be part of any of the
following categories:
\begin{itemize}
  \item \textbf{Core developers (HERON core team)}: These are the developers of the \textbf{\textit{HERON}}  plug-in. They will be responsible for following
    and enforcing the appropriate software development standards. They will be responsible for designing, implementing and 
    maintaining the plug-in.
  \item \textbf{External developers}: A Scientist or Engineer that utilizes the \textbf{\textit{HERON}}  plug-in and wants to extend its 
  capabilities.This user will typically have a background in modeling and 
simulation techniques and/or numerical and economic analysis but may only have a limited skill-set when it comes to object-oriented 
coding, C++/Python languages.
  \item \textbf{Analysts}:  These are users that will run the plug-in (in conjunction with RAVEN) and perform various analysis on the 
  simulations they perform. These users may interact with developers of the system requesting new features and reporting bugs found 
  and will typically make heavy use of the input file format.
\end{itemize}
