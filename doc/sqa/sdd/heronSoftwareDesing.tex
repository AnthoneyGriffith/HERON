\section{Software Design}
\subsection{Introduction}
The \textit{\textbf{HERON}} (Hollistic Energy Resource Optimization Network) is a project to construct workflows solving
complex optimization of component sizing and stochastic resource allocation to meet target economic goals.
The \textit{\textbf{HERON}} plug-in has been coded using the language \emph{Python}. The \emph{Python}
 code provides ``external model'' in RAVEN (for installation and usage instructions, see~\cite{RAVENuserManual}) as well as
RAVEN  workflow generation tool.
The input of  \textit{\textbf{HERON}} is an XML file, which can be read and executed by the plug-in and RAVEN.


\subsection{System Structure (Code)}
The  \textit{\textbf{HERON}} plug-in is based on the RAVEN plug-in system. This system is aimed to ease the creation
of RAVEN entities, workflows and modules by external developers without the need to deeply know the internal structure
of the RAVEN software. This system is a transparent API for RAVEN.
\\The addition of a plugin does not require modifying RAVEN itself. 
Instead, the developer creates a new Python module that is going to be embedded
 in RAVEN at run-time (no need to introduce  hard-coded statements).


In order to install the HERON plugin (if not downloaded by RAVEN submodule system),
 the user can run the script contained in the RAVEN script folder:
\begin{lstlisting}[language=bash]
 python path/to/raven/scripts/install_plugins.py  -s **directory**/Heron
\end{lstlisting}
where  $**directory**$/Heron should be replaced with the absolute path to the Heron plugin directory.
(e.g. ``path/to/my/plugins/Heron''). 

\subsection{HERON Structure}
The  \textit{\textbf{HERON}} plug-in contains the following methods (API from RAVEN):

\begin{lstlisting}[language=python]
from InputTemplates.TemplateBaseClass import Template as TemplateBase

class Template(TemplateBase):
  def run (self, container, Inputs)
\end{lstlisting}
In the following sub-sections all the methods are explained.
\subsubsection{Method: \texttt{run}}
\label{subsubsec:runExternalModelPlugin}
\begin{lstlisting}[language=python]
def run (self, container, Inputs)
\end{lstlisting}

In this function, the HERON analysis is coded.
%
The only two attributes this method is going to receive are a Python list of inputs
(the inputs coming from the \texttt{createNewInput} method (see \cite{RAVENuserManual}) and a ``self-like'' object
named ``container''.
%
All the outcomes of the HERON module will be stored in the above mentioned ``container'' in order to 
allow RAVEN to collect them.

Example XML:
\begin{lstlisting}[style=XML,morekeywords={subType,ModuleToLoad}]
  <Components>
    <Component name='PowerPlant1'>
      <produces resource='electricity' dispatchable='True'>
        <capacity>
          <sweep_values>80, 90, 100</sweep_values>
          <opt_bounds>80, 100</opt_bounds>
        </capacity>
        <transfer>
          <fixed_value>10</fixed_value>
        </transfer>
      </produces>
      <economics>
        <CashFlow name='capex' type='one-time' taxable='False' inflation='none' mult_target='False'>
          <driver>capacity</driver>
          <reference_price>
            <ARMA variable="e_price">price_data</ARMA>
          </reference_price>
          <reference_driver>1e3</reference_driver>
          <scaling_factor_x>1.0</scaling_factor_x>
        </CashFlow>
      </economics>
    </Component>
  </Components>
\end{lstlisting}

  \begin{itemize}
    \item \texttt{Components} -- Represents a unit in the grid analysis. Each component has a single "interaction" that
    describes what it can do (produce, store, demand)
    \item \texttt{produces} -- Explains a particular interaction, where a resource is consumed to produce another resource.
    \item \texttt{stores} -- Explains a particular interaction, where a resource is stored and released later.
    \item \texttt{demands} -- Explains a particular interaction, where a resource is consumed.
    \item \texttt{capacity} -- capacity of the interaction of this component.
    \item \texttt{transfers} -- implements transfer logic for component operation.
    \item \texttt{economics} -- Defines the Economics Capability of the component.
    \item \texttt{CashFlow} -- is a CashFlow entity from RAVEN plug-in CashFlow.
    \begin{itemize}
    \item \texttt{name} -- Identifier of CashFlow.
    \item \texttt{type} -- Describe cyclic of the CashFlow Economics.
    \item \texttt{taxable} -- Whether to apply Tax or not.
    \item \texttt{inflation} -- Whether to apply Inflation or not.
    \end{itemize}  
    \item \texttt{driver} -- quantity produced
  \end{itemize}




