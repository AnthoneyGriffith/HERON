%
% This is an example LaTeX file which uses the SANDreport class file.
% It shows how a SAND report should be formatted, what sections and
% elements it should contain, and how to use the SANDreport class.
% It uses the LaTeX article class, but not the strict option.
% ItINLreport uses .eps logos and files to show how pdflatex can be used
%
% Get the latest version of the class file and more at
%    http://www.cs.sandia.gov/~rolf/SANDreport
%
% This file and the SANDreport.cls file are based on information
% contained in "Guide to Preparing {SAND} Reports", Sand98-0730, edited
% by Tamara K. Locke, and the newer "Guide to Preparing SAND Reports and
% Other Communication Products", SAND2002-2068P.
% Please send corrections and suggestions for improvements to
% Rolf Riesen, Org. 9223, MS 1110, rolf@cs.sandia.gov
%
\documentclass[pdf,12pt]{INLreport}
%\documentclass[12pt]{article}
% pslatex is really old (1994).  It attempts to merge the times and mathptm packages.
% My opinion is that it produces a really bad looking math font.  So why are we using it?
% If you just want to change the text font, you should just \usepackage{times}.
% \usepackage{pslatex}
\usepackage{times}
\usepackage{longtable}
\usepackage[FIGBOTCAP,normal,bf,tight]{subfigure}
\usepackage{amsmath}
\usepackage{amssymb}
\usepackage[labelfont=bf]{caption}
\usepackage{pifont}
\usepackage{enumerate}
\usepackage{listings}
\usepackage{fullpage}
\usepackage{xcolor}          % Using xcolor for more robust color specification
\usepackage{ifthen}          % For simple checking in newcommand blocks
\usepackage{textcomp}
%\usepackage{authblk}         % For making the author list look prettier
%\renewcommand\Authsep{,~\,}

% Custom colors
\definecolor{deepblue}{rgb}{0,0,0.5}
\definecolor{deepred}{rgb}{0.6,0,0}
\definecolor{deepgreen}{rgb}{0,0.5,0}
\definecolor{forestgreen}{RGB}{34,139,34}
\definecolor{orangered}{RGB}{239,134,64}
\definecolor{darkblue}{rgb}{0.0,0.0,0.6}
\definecolor{gray}{rgb}{0.4,0.4,0.4}

\lstset {
  basicstyle=\ttfamily,
  frame=single
}

\setcounter{secnumdepth}{5}
\lstdefinestyle{XML} {
    language=XML,
    extendedchars=true,
    breaklines=true,
    breakatwhitespace=true,
%    emph={name,dim,interactive,overwrite},
    emphstyle=\color{red},
    basicstyle=\ttfamily,
%    columns=fullflexible,
    commentstyle=\color{gray}\upshape,
    morestring=[b]",
    morecomment=[s]{<?}{?>},
    morecomment=[s][\color{forestgreen}]{<!--}{-->},
    keywordstyle=\color{cyan},
    stringstyle=\ttfamily\color{black},
    tagstyle=\color{darkblue}\bf\ttfamily,
    morekeywords={name,type},
%    morekeywords={name,attribute,source,variables,version,type,release,x,z,y,xlabel,ylabel,how,text,param1,param2,color,label},
}
\lstset{language=python,upquote=true}

\usepackage{titlesec}
\newcommand{\sectionbreak}{\clearpage}
\setcounter{secnumdepth}{4}

%\titleformat{\paragraph}
%{\normalfont\normalsize\bfseries}{\theparagraph}{1em}{}
%\titlespacing*{\paragraph}
%{0pt}{3.25ex plus 1ex minus .2ex}{1.5ex plus .2ex}

%%%%%%%% Begin comands definition to input python code into document
\usepackage[utf8]{inputenc}

% Default fixed font does not support bold face
\DeclareFixedFont{\ttb}{T1}{txtt}{bx}{n}{9} % for bold
\DeclareFixedFont{\ttm}{T1}{txtt}{m}{n}{9}  % for normal

\usepackage{listings}

% Python style for highlighting
\newcommand\pythonstyle{\lstset{
language=Python,
basicstyle=\ttm,
otherkeywords={self, none, return},             % Add keywords here
keywordstyle=\ttb\color{deepblue},
emph={MyClass,__init__},          % Custom highlighting
emphstyle=\ttb\color{deepred},    % Custom highlighting style
stringstyle=\color{deepgreen},
frame=tb,                         % Any extra options here
showstringspaces=false            %
}}


% Python environment
\lstnewenvironment{python}[1][]
{
\pythonstyle
\lstset{#1}
}
{}

% Python for external files
\newcommand\pythonexternal[2][]{{
\pythonstyle
\lstinputlisting[#1]{#2}}}

\lstnewenvironment{xml}
{}
{}

% Python for inline
\newcommand\pythoninline[1]{{\pythonstyle\lstinline!#1!}}

% Named Colors for the comments below (Attempted to match git symbol colors)
\definecolor{RScolor}{HTML}{8EB361}  % Sonat (adjusted for clarity)
\definecolor{DPMcolor}{HTML}{E28B8D} % Dan
\definecolor{JCcolor}{HTML}{82A8D9}  % Josh (adjusted for clarity)
\definecolor{AAcolor}{HTML}{8D7F44}  % Andrea
\definecolor{CRcolor}{HTML}{AC39CE}  % Cristian
\definecolor{RKcolor}{HTML}{3ECC8D}  % Bob (adjusted for clarity)
\definecolor{DMcolor}{HTML}{276605}  % Diego (adjusted for clarity)
\definecolor{PTcolor}{HTML}{990000}  % Paul

\def\DRAFT{} % Uncomment this if you want to see the notes people have been adding
% Comment command for developers (Should only be used under active development)
\ifdefined\DRAFT
  \newcommand{\nameLabeler}[3]{\textcolor{#2}{[[#1: #3]]}}
\else
  \newcommand{\nameLabeler}[3]{}
\fi
\newcommand{\alfoa}[1] {\nameLabeler{Andrea}{AAcolor}{#1}}
\newcommand{\cristr}[1] {\nameLabeler{Cristian}{CRcolor}{#1}}
%\newcommand{\mandd}[1] {\nameLabeler{Diego}{DMcolor}{#1}}
%\newcommand{\maljdan}[1] {\nameLabeler{Dan}{DPMcolor}{#1}}
%\newcommand{\cogljj}[1] {\nameLabeler{Josh}{JCcolor}{#1}}
%\newcommand{\bobk}[1] {\nameLabeler{Bob}{RKcolor}{#1}}
%\newcommand{\senrs}[1] {\nameLabeler{Sonat}{RScolor}{#1}}
\newcommand{\talbpaul}[1] {\nameLabeler{Paul}{PTcolor}{#1}}
% Commands for making the LaTeX a bit more uniform and cleaner
\newcommand{\TODO}[1]    {\textcolor{red}{\textit{(#1)}}}
\newcommand{\xmlAttrRequired}[1] {\textcolor{red}{\textbf{\texttt{#1}}}}
\newcommand{\xmlAttr}[1] {\textcolor{cyan}{\textbf{\texttt{#1}}}}
\newcommand{\xmlNodeRequired}[1] {\textcolor{deepblue}{\textbf{\texttt{<#1>}}}}
\newcommand{\xmlNode}[1] {\textcolor{darkblue}{\textbf{\texttt{<#1>}}}}
\newcommand{\xmlString}[1] {\textcolor{black}{\textbf{\texttt{'#1'}}}}
\newcommand{\xmlDesc}[1] {\textbf{\textit{#1}}} % Maybe a misnomer, but I am
                                                % using this to detail the data
                                                % type and necessity of an XML
                                                % node or attribute,
                                                % xmlDesc = XML description
\newcommand{\default}[1]{~\\*\textit{Default: #1}}
\newcommand{\nb} {\textcolor{deepgreen}{\textbf{~Note:}}~}

\usepackage{bm}

\newcommand{\tensor}[1]{{\bm{#1}}}

\renewcommand{\vec}{\bm}


\newcommand{\unit}[1]{\hat{\bm{#1}}}



\newcommand{\scalarunit}[1]{\hat{#1}}


\usepackage{booktabs}

\usepackage{stmaryrd}

\usepackage{hyperref}
\hypersetup{
    colorlinks,
    citecolor=black,
    filecolor=black,
    linkcolor=black,
    urlcolor=black
}

\newcommand{\wiki}{\href{https://github.com/idaholab/raven/wiki}{RAVEN wiki}}

\usepackage{cite}


\raggedbottom
\setcounter{secnumdepth}{5} % show 5 levels of subsection
\setcounter{tocdepth}{5} % include 5 levels of subsection in table of contents


\title{HERON User Manual}


\author{
\textbf{\textit{Project Manager:}}
 \\Cristian Rabiti\\
 \textbf{\textit{Principal Investigator and Technical Leader:}}
\\Paul W. Talbot\\
\textbf{\textit{Main Developers:}}
\\Paul W. Talbot
\\Andrea Alfonsi
%\\Diego Mandelli
%\\Joshua Cogliati
%\\Congjian Wang
%\\Daniel P. Maljovec
%\\Robert Kinoshita\\
%\textbf{\textit{Former Developers:}} \\Sonat Sen
%\\Jun Chen\\
%\textbf{\textit{Contributors:}}
%\\Alessandro Bandini (Post-Processor)
%\\Ivan Rinaldi (documentation)
%\\Claudia Picoco (new external code interface)
%\\James B. Tompkins (new external code interface)
%\\Matteo Donorio (new external code interface)
%\\Fabio Giannetti (new external code interface)
%\\Jia Zhou (conjugate gradient optimizer)
}

\date{}



\SANDnum{INL/EXT-15-34123}
\SANDprintDate{\today}
\SANDauthor{Cristian Rabiti, Paul W. Talbot, Andrea Alfonsi}
\SANDreleaseType{Revision 1}



\def\component#1{\texttt{#1}}

% ---------------------------------------------------------------------------- %
\newcommand{\systemtau}{\tensor{\tau}_{\!\text{SUPG}}}

% Added by Sonat
\usepackage{placeins}
\usepackage{array}

\newcolumntype{L}[1]{>{\raggedright\let\newline\\\arraybackslash\hspace{0pt}}m{#1}}
\newcolumntype{C}[1]{>{\centering\let\newline\\\arraybackslash\hspace{0pt}}m{#1}}
\newcolumntype{R}[1]{>{\raggedleft\let\newline\\\arraybackslash\hspace{0pt}}m{#1}}



%\begin{document}
 %   \maketitle


%    \SANDmain		

%\input{introduction.tex}
%\input{nomenclature.tex}
%\input{Installation/main.tex}
%\input{HowToRun.tex}
%\input{HERONStructure.tex}
%\input{runInfo.tex}
%\input{files.tex}
%\input{examplesPrimer.tex}
%\section*{Document Version Information}
%\input{../version.tex}



%    \clearpage

%    \providecommand*{\phantomsection}{}
%    \phantomsection
%    \addcontentsline{toc}{section}{References}
%    \bibliographystyle{ieeetr}
%    \bibliography{raven_user_manual}


%\end{document}
 \begin{document}
    \maketitle
    \SANDmain
    \section{Cases Introduction}Lorem ipsum dolor sit amet, consectetur adipiscing elit, sed do eiusmod tempor incididunt ut labore et dolore magna aliqua.



\subsection{Case}
  -- no description yet --

  The \xmlNode{Case} node recognizes the following parameters:
    \begin{itemize}
      \item \xmlAttr{name}: \xmlDesc{string, required}, 
        -- no description yet --
  \end{itemize}

  The \xmlNode{Case} node recognizes the following subnodes:
  \begin{itemize}
    \item \xmlNode{mode}:\xmlDesc{ModeOptions}, 
      -- no description yet --

    \item \xmlNode{metric}:\xmlDesc{EconMetrics}, 
      -- no description yet --

    \item \xmlNode{differential}:\xmlDesc{bool}, 
      -- no description yet --

    \item \xmlNode{num\_arma\_samples}:\xmlDesc{integer}, 
      -- no description yet --

    \item \xmlNode{timestep\_interval}:\xmlDesc{integer}, 
      -- no description yet --

    \item \xmlNode{history\_length}:\xmlDesc{integer}, 
      -- no description yet --

    \item \xmlNode{economics}:
      -- no description yet --

      The \xmlNode{economics} node recognizes the following subnodes:
      \begin{itemize}
        \item \xmlNode{ProjectTime}:\xmlDesc{float}, 
          -- no description yet --

        \item \xmlNode{DiscountRate}:\xmlDesc{float}, 
          -- no description yet --

        \item \xmlNode{tax}:\xmlDesc{float}, 
          -- no description yet --

        \item \xmlNode{inflation}:\xmlDesc{float}, 
          -- no description yet --

        \item \xmlNode{verbosity}:\xmlDesc{integer}, 
          -- no description yet --
      \end{itemize}

    \item \xmlNode{dispatch\_increment}:\xmlDesc{float}, 
      -- no description yet --
      The \xmlNode{dispatch\_increment} node recognizes the following parameters:
        \begin{itemize}
          \item \xmlAttr{resource}: \xmlDesc{string, required}, 
            -- no description yet --
      \end{itemize}
  \end{itemize}

\section{Components Introduction}Lorem ipsum dolor sit amet, consectetur adipiscing elit, sed do eiusmod tempor incididunt ut labore et dolore magna aliqua.



\subsection{Component}
  -- no description yet --

  The \xmlNode{Component} node recognizes the following parameters:
    \begin{itemize}
      \item \xmlAttr{name}: \xmlDesc{string, required}, 
        -- no description yet --
  \end{itemize}

  The \xmlNode{Component} node recognizes the following subnodes:
  \begin{itemize}
    \item \xmlNode{produces}:
      -- no description yet --
      The \xmlNode{produces} node recognizes the following parameters:
        \begin{itemize}
          \item \xmlAttr{resource}: \xmlDesc{string\_list, required}, 
            -- no description yet --
          \item \xmlAttr{dispatch}: \xmlDesc{dispatch\_opts, optional}, 
            -- no description yet --
      \end{itemize}

      The \xmlNode{produces} node recognizes the following subnodes:
      \begin{itemize}
        \item \xmlNode{capacity}:
          -- no description yet --
          The \xmlNode{capacity} node recognizes the following parameters:
            \begin{itemize}
              \item \xmlAttr{resource}: \xmlDesc{string, optional}, 
                -- no description yet --
          \end{itemize}

          The \xmlNode{capacity} node recognizes the following subnodes:
          \begin{itemize}
            \item \xmlNode{fixed\_value}:\xmlDesc{float}, 
              -- no description yet --

            \item \xmlNode{sweep\_values}:\xmlDesc{float\_list}, 
              -- no description yet --

            \item \xmlNode{opt\_bounds}:\xmlDesc{float\_list}, 
              -- no description yet --

            \item \xmlNode{ARMA}:\xmlDesc{string}, 
              -- no description yet --
              The \xmlNode{ARMA} node recognizes the following parameters:
                \begin{itemize}
                  \item \xmlAttr{variable}: \xmlDesc{string, optional}, 
                    -- no description yet --
              \end{itemize}

            \item \xmlNode{Function}:\xmlDesc{string}, 
              -- no description yet --
              The \xmlNode{Function} node recognizes the following parameters:
                \begin{itemize}
                  \item \xmlAttr{method}: \xmlDesc{string, optional}, 
                    -- no description yet --
              \end{itemize}

            \item \xmlNode{variable}:\xmlDesc{string}, 
              -- no description yet --

            \item \xmlNode{growth}:\xmlDesc{float}, 
              -- no description yet --
              The \xmlNode{growth} node recognizes the following parameters:
                \begin{itemize}
                  \item \xmlAttr{mode}: \xmlDesc{growthType, optional}, 
                    -- no description yet --
              \end{itemize}
          \end{itemize}

        \item \xmlNode{minimum}:
          -- no description yet --
          The \xmlNode{minimum} node recognizes the following parameters:
            \begin{itemize}
              \item \xmlAttr{resource}: \xmlDesc{string, optional}, 
                -- no description yet --
          \end{itemize}

          The \xmlNode{minimum} node recognizes the following subnodes:
          \begin{itemize}
            \item \xmlNode{fixed\_value}:\xmlDesc{float}, 
              -- no description yet --

            \item \xmlNode{sweep\_values}:\xmlDesc{float\_list}, 
              -- no description yet --

            \item \xmlNode{opt\_bounds}:\xmlDesc{float\_list}, 
              -- no description yet --

            \item \xmlNode{ARMA}:\xmlDesc{string}, 
              -- no description yet --
              The \xmlNode{ARMA} node recognizes the following parameters:
                \begin{itemize}
                  \item \xmlAttr{variable}: \xmlDesc{string, optional}, 
                    -- no description yet --
              \end{itemize}

            \item \xmlNode{Function}:\xmlDesc{string}, 
              -- no description yet --
              The \xmlNode{Function} node recognizes the following parameters:
                \begin{itemize}
                  \item \xmlAttr{method}: \xmlDesc{string, optional}, 
                    -- no description yet --
              \end{itemize}

            \item \xmlNode{variable}:\xmlDesc{string}, 
              -- no description yet --

            \item \xmlNode{growth}:\xmlDesc{float}, 
              -- no description yet --
              The \xmlNode{growth} node recognizes the following parameters:
                \begin{itemize}
                  \item \xmlAttr{mode}: \xmlDesc{growthType, optional}, 
                    -- no description yet --
              \end{itemize}
          \end{itemize}

        \item \xmlNode{consumes}:\xmlDesc{string\_list}, 
          -- no description yet --

        \item \xmlNode{transfer}:
          -- no description yet --

          The \xmlNode{transfer} node recognizes the following subnodes:
          \begin{itemize}
            \item \xmlNode{fixed\_value}:\xmlDesc{float}, 
              -- no description yet --

            \item \xmlNode{sweep\_values}:\xmlDesc{float\_list}, 
              -- no description yet --

            \item \xmlNode{opt\_bounds}:\xmlDesc{float\_list}, 
              -- no description yet --

            \item \xmlNode{ARMA}:\xmlDesc{string}, 
              -- no description yet --
              The \xmlNode{ARMA} node recognizes the following parameters:
                \begin{itemize}
                  \item \xmlAttr{variable}: \xmlDesc{string, optional}, 
                    -- no description yet --
              \end{itemize}

            \item \xmlNode{Function}:\xmlDesc{string}, 
              -- no description yet --
              The \xmlNode{Function} node recognizes the following parameters:
                \begin{itemize}
                  \item \xmlAttr{method}: \xmlDesc{string, optional}, 
                    -- no description yet --
              \end{itemize}

            \item \xmlNode{variable}:\xmlDesc{string}, 
              -- no description yet --

            \item \xmlNode{growth}:\xmlDesc{float}, 
              -- no description yet --
              The \xmlNode{growth} node recognizes the following parameters:
                \begin{itemize}
                  \item \xmlAttr{mode}: \xmlDesc{growthType, optional}, 
                    -- no description yet --
              \end{itemize}
          \end{itemize}
      \end{itemize}

    \item \xmlNode{stores}:
      -- no description yet --
      The \xmlNode{stores} node recognizes the following parameters:
        \begin{itemize}
          \item \xmlAttr{resource}: \xmlDesc{string\_list, required}, 
            -- no description yet --
          \item \xmlAttr{dispatch}: \xmlDesc{dispatch\_opts, optional}, 
            -- no description yet --
      \end{itemize}

      The \xmlNode{stores} node recognizes the following subnodes:
      \begin{itemize}
        \item \xmlNode{capacity}:
          -- no description yet --
          The \xmlNode{capacity} node recognizes the following parameters:
            \begin{itemize}
              \item \xmlAttr{resource}: \xmlDesc{string, optional}, 
                -- no description yet --
          \end{itemize}

          The \xmlNode{capacity} node recognizes the following subnodes:
          \begin{itemize}
            \item \xmlNode{fixed\_value}:\xmlDesc{float}, 
              -- no description yet --

            \item \xmlNode{sweep\_values}:\xmlDesc{float\_list}, 
              -- no description yet --

            \item \xmlNode{opt\_bounds}:\xmlDesc{float\_list}, 
              -- no description yet --

            \item \xmlNode{ARMA}:\xmlDesc{string}, 
              -- no description yet --
              The \xmlNode{ARMA} node recognizes the following parameters:
                \begin{itemize}
                  \item \xmlAttr{variable}: \xmlDesc{string, optional}, 
                    -- no description yet --
              \end{itemize}

            \item \xmlNode{Function}:\xmlDesc{string}, 
              -- no description yet --
              The \xmlNode{Function} node recognizes the following parameters:
                \begin{itemize}
                  \item \xmlAttr{method}: \xmlDesc{string, optional}, 
                    -- no description yet --
              \end{itemize}

            \item \xmlNode{variable}:\xmlDesc{string}, 
              -- no description yet --

            \item \xmlNode{growth}:\xmlDesc{float}, 
              -- no description yet --
              The \xmlNode{growth} node recognizes the following parameters:
                \begin{itemize}
                  \item \xmlAttr{mode}: \xmlDesc{growthType, optional}, 
                    -- no description yet --
              \end{itemize}
          \end{itemize}

        \item \xmlNode{minimum}:
          -- no description yet --
          The \xmlNode{minimum} node recognizes the following parameters:
            \begin{itemize}
              \item \xmlAttr{resource}: \xmlDesc{string, optional}, 
                -- no description yet --
          \end{itemize}

          The \xmlNode{minimum} node recognizes the following subnodes:
          \begin{itemize}
            \item \xmlNode{fixed\_value}:\xmlDesc{float}, 
              -- no description yet --

            \item \xmlNode{sweep\_values}:\xmlDesc{float\_list}, 
              -- no description yet --

            \item \xmlNode{opt\_bounds}:\xmlDesc{float\_list}, 
              -- no description yet --

            \item \xmlNode{ARMA}:\xmlDesc{string}, 
              -- no description yet --
              The \xmlNode{ARMA} node recognizes the following parameters:
                \begin{itemize}
                  \item \xmlAttr{variable}: \xmlDesc{string, optional}, 
                    -- no description yet --
              \end{itemize}

            \item \xmlNode{Function}:\xmlDesc{string}, 
              -- no description yet --
              The \xmlNode{Function} node recognizes the following parameters:
                \begin{itemize}
                  \item \xmlAttr{method}: \xmlDesc{string, optional}, 
                    -- no description yet --
              \end{itemize}

            \item \xmlNode{variable}:\xmlDesc{string}, 
              -- no description yet --

            \item \xmlNode{growth}:\xmlDesc{float}, 
              -- no description yet --
              The \xmlNode{growth} node recognizes the following parameters:
                \begin{itemize}
                  \item \xmlAttr{mode}: \xmlDesc{growthType, optional}, 
                    -- no description yet --
              \end{itemize}
          \end{itemize}

        \item \xmlNode{rate}:
          -- no description yet --

          The \xmlNode{rate} node recognizes the following subnodes:
          \begin{itemize}
            \item \xmlNode{fixed\_value}:\xmlDesc{float}, 
              -- no description yet --

            \item \xmlNode{sweep\_values}:\xmlDesc{float\_list}, 
              -- no description yet --

            \item \xmlNode{opt\_bounds}:\xmlDesc{float\_list}, 
              -- no description yet --

            \item \xmlNode{ARMA}:\xmlDesc{string}, 
              -- no description yet --
              The \xmlNode{ARMA} node recognizes the following parameters:
                \begin{itemize}
                  \item \xmlAttr{variable}: \xmlDesc{string, optional}, 
                    -- no description yet --
              \end{itemize}

            \item \xmlNode{Function}:\xmlDesc{string}, 
              -- no description yet --
              The \xmlNode{Function} node recognizes the following parameters:
                \begin{itemize}
                  \item \xmlAttr{method}: \xmlDesc{string, optional}, 
                    -- no description yet --
              \end{itemize}

            \item \xmlNode{variable}:\xmlDesc{string}, 
              -- no description yet --

            \item \xmlNode{growth}:\xmlDesc{float}, 
              -- no description yet --
              The \xmlNode{growth} node recognizes the following parameters:
                \begin{itemize}
                  \item \xmlAttr{mode}: \xmlDesc{growthType, optional}, 
                    -- no description yet --
              \end{itemize}
          \end{itemize}

        \item \xmlNode{initial\_stored}:
          -- no description yet --

          The \xmlNode{initial\_stored} node recognizes the following subnodes:
          \begin{itemize}
            \item \xmlNode{fixed\_value}:\xmlDesc{float}, 
              -- no description yet --

            \item \xmlNode{sweep\_values}:\xmlDesc{float\_list}, 
              -- no description yet --

            \item \xmlNode{opt\_bounds}:\xmlDesc{float\_list}, 
              -- no description yet --

            \item \xmlNode{ARMA}:\xmlDesc{string}, 
              -- no description yet --
              The \xmlNode{ARMA} node recognizes the following parameters:
                \begin{itemize}
                  \item \xmlAttr{variable}: \xmlDesc{string, optional}, 
                    -- no description yet --
              \end{itemize}

            \item \xmlNode{Function}:\xmlDesc{string}, 
              -- no description yet --
              The \xmlNode{Function} node recognizes the following parameters:
                \begin{itemize}
                  \item \xmlAttr{method}: \xmlDesc{string, optional}, 
                    -- no description yet --
              \end{itemize}

            \item \xmlNode{variable}:\xmlDesc{string}, 
              -- no description yet --

            \item \xmlNode{growth}:\xmlDesc{float}, 
              -- no description yet --
              The \xmlNode{growth} node recognizes the following parameters:
                \begin{itemize}
                  \item \xmlAttr{mode}: \xmlDesc{growthType, optional}, 
                    -- no description yet --
              \end{itemize}
          \end{itemize}
      \end{itemize}

    \item \xmlNode{demands}:
      -- no description yet --
      The \xmlNode{demands} node recognizes the following parameters:
        \begin{itemize}
          \item \xmlAttr{resource}: \xmlDesc{string\_list, required}, 
            -- no description yet --
          \item \xmlAttr{dispatch}: \xmlDesc{dispatch\_opts, optional}, 
            -- no description yet --
      \end{itemize}

      The \xmlNode{demands} node recognizes the following subnodes:
      \begin{itemize}
        \item \xmlNode{capacity}:
          -- no description yet --
          The \xmlNode{capacity} node recognizes the following parameters:
            \begin{itemize}
              \item \xmlAttr{resource}: \xmlDesc{string, optional}, 
                -- no description yet --
          \end{itemize}

          The \xmlNode{capacity} node recognizes the following subnodes:
          \begin{itemize}
            \item \xmlNode{fixed\_value}:\xmlDesc{float}, 
              -- no description yet --

            \item \xmlNode{sweep\_values}:\xmlDesc{float\_list}, 
              -- no description yet --

            \item \xmlNode{opt\_bounds}:\xmlDesc{float\_list}, 
              -- no description yet --

            \item \xmlNode{ARMA}:\xmlDesc{string}, 
              -- no description yet --
              The \xmlNode{ARMA} node recognizes the following parameters:
                \begin{itemize}
                  \item \xmlAttr{variable}: \xmlDesc{string, optional}, 
                    -- no description yet --
              \end{itemize}

            \item \xmlNode{Function}:\xmlDesc{string}, 
              -- no description yet --
              The \xmlNode{Function} node recognizes the following parameters:
                \begin{itemize}
                  \item \xmlAttr{method}: \xmlDesc{string, optional}, 
                    -- no description yet --
              \end{itemize}

            \item \xmlNode{variable}:\xmlDesc{string}, 
              -- no description yet --

            \item \xmlNode{growth}:\xmlDesc{float}, 
              -- no description yet --
              The \xmlNode{growth} node recognizes the following parameters:
                \begin{itemize}
                  \item \xmlAttr{mode}: \xmlDesc{growthType, optional}, 
                    -- no description yet --
              \end{itemize}
          \end{itemize}

        \item \xmlNode{minimum}:
          -- no description yet --
          The \xmlNode{minimum} node recognizes the following parameters:
            \begin{itemize}
              \item \xmlAttr{resource}: \xmlDesc{string, optional}, 
                -- no description yet --
          \end{itemize}

          The \xmlNode{minimum} node recognizes the following subnodes:
          \begin{itemize}
            \item \xmlNode{fixed\_value}:\xmlDesc{float}, 
              -- no description yet --

            \item \xmlNode{sweep\_values}:\xmlDesc{float\_list}, 
              -- no description yet --

            \item \xmlNode{opt\_bounds}:\xmlDesc{float\_list}, 
              -- no description yet --

            \item \xmlNode{ARMA}:\xmlDesc{string}, 
              -- no description yet --
              The \xmlNode{ARMA} node recognizes the following parameters:
                \begin{itemize}
                  \item \xmlAttr{variable}: \xmlDesc{string, optional}, 
                    -- no description yet --
              \end{itemize}

            \item \xmlNode{Function}:\xmlDesc{string}, 
              -- no description yet --
              The \xmlNode{Function} node recognizes the following parameters:
                \begin{itemize}
                  \item \xmlAttr{method}: \xmlDesc{string, optional}, 
                    -- no description yet --
              \end{itemize}

            \item \xmlNode{variable}:\xmlDesc{string}, 
              -- no description yet --

            \item \xmlNode{growth}:\xmlDesc{float}, 
              -- no description yet --
              The \xmlNode{growth} node recognizes the following parameters:
                \begin{itemize}
                  \item \xmlAttr{mode}: \xmlDesc{growthType, optional}, 
                    -- no description yet --
              \end{itemize}
          \end{itemize}

        \item \xmlNode{penalty}:
          -- no description yet --

          The \xmlNode{penalty} node recognizes the following subnodes:
          \begin{itemize}
            \item \xmlNode{fixed\_value}:\xmlDesc{float}, 
              -- no description yet --

            \item \xmlNode{sweep\_values}:\xmlDesc{float\_list}, 
              -- no description yet --

            \item \xmlNode{opt\_bounds}:\xmlDesc{float\_list}, 
              -- no description yet --

            \item \xmlNode{ARMA}:\xmlDesc{string}, 
              -- no description yet --
              The \xmlNode{ARMA} node recognizes the following parameters:
                \begin{itemize}
                  \item \xmlAttr{variable}: \xmlDesc{string, optional}, 
                    -- no description yet --
              \end{itemize}

            \item \xmlNode{Function}:\xmlDesc{string}, 
              -- no description yet --
              The \xmlNode{Function} node recognizes the following parameters:
                \begin{itemize}
                  \item \xmlAttr{method}: \xmlDesc{string, optional}, 
                    -- no description yet --
              \end{itemize}

            \item \xmlNode{variable}:\xmlDesc{string}, 
              -- no description yet --

            \item \xmlNode{growth}:\xmlDesc{float}, 
              -- no description yet --
              The \xmlNode{growth} node recognizes the following parameters:
                \begin{itemize}
                  \item \xmlAttr{mode}: \xmlDesc{growthType, optional}, 
                    -- no description yet --
              \end{itemize}
          \end{itemize}
      \end{itemize}

    \item \xmlNode{economics}:
      -- no description yet --

      The \xmlNode{economics} node recognizes the following subnodes:
      \begin{itemize}
        \item \xmlNode{lifetime}:\xmlDesc{integer}, 
          -- no description yet --

        \item \xmlNode{CashFlow}:
          -- no description yet --
          The \xmlNode{CashFlow} node recognizes the following parameters:
            \begin{itemize}
              \item \xmlAttr{name}: \xmlDesc{string, required}, 
                -- no description yet --
              \item \xmlAttr{type}: \xmlDesc{string, required}, 
                -- no description yet --
              \item \xmlAttr{taxable}: \xmlDesc{bool, required}, 
                -- no description yet --
              \item \xmlAttr{inflation}: \xmlDesc{string, required}, 
                -- no description yet --
              \item \xmlAttr{mult\_target}: \xmlDesc{bool, required}, 
                -- no description yet --
              \item \xmlAttr{period}: \xmlDesc{period\_opts, optional}, 
                -- no description yet --
          \end{itemize}

          The \xmlNode{CashFlow} node recognizes the following subnodes:
          \begin{itemize}
            \item \xmlNode{driver}:
              -- no description yet --

              The \xmlNode{driver} node recognizes the following subnodes:
              \begin{itemize}
                \item \xmlNode{fixed\_value}:\xmlDesc{float}, 
                  -- no description yet --

                \item \xmlNode{sweep\_values}:\xmlDesc{float\_list}, 
                  -- no description yet --

                \item \xmlNode{opt\_bounds}:\xmlDesc{float\_list}, 
                  -- no description yet --

                \item \xmlNode{ARMA}:\xmlDesc{string}, 
                  -- no description yet --
                  The \xmlNode{ARMA} node recognizes the following parameters:
                    \begin{itemize}
                      \item \xmlAttr{variable}: \xmlDesc{string, optional}, 
                        -- no description yet --
                  \end{itemize}

                \item \xmlNode{Function}:\xmlDesc{string}, 
                  -- no description yet --
                  The \xmlNode{Function} node recognizes the following parameters:
                    \begin{itemize}
                      \item \xmlAttr{method}: \xmlDesc{string, optional}, 
                        -- no description yet --
                  \end{itemize}

                \item \xmlNode{variable}:\xmlDesc{string}, 
                  -- no description yet --

                \item \xmlNode{growth}:\xmlDesc{float}, 
                  -- no description yet --
                  The \xmlNode{growth} node recognizes the following parameters:
                    \begin{itemize}
                      \item \xmlAttr{mode}: \xmlDesc{growthType, optional}, 
                        -- no description yet --
                  \end{itemize}
              \end{itemize}

            \item \xmlNode{reference\_price}:
              -- no description yet --

              The \xmlNode{reference\_price} node recognizes the following subnodes:
              \begin{itemize}
                \item \xmlNode{fixed\_value}:\xmlDesc{float}, 
                  -- no description yet --

                \item \xmlNode{sweep\_values}:\xmlDesc{float\_list}, 
                  -- no description yet --

                \item \xmlNode{opt\_bounds}:\xmlDesc{float\_list}, 
                  -- no description yet --

                \item \xmlNode{ARMA}:\xmlDesc{string}, 
                  -- no description yet --
                  The \xmlNode{ARMA} node recognizes the following parameters:
                    \begin{itemize}
                      \item \xmlAttr{variable}: \xmlDesc{string, optional}, 
                        -- no description yet --
                  \end{itemize}

                \item \xmlNode{Function}:\xmlDesc{string}, 
                  -- no description yet --
                  The \xmlNode{Function} node recognizes the following parameters:
                    \begin{itemize}
                      \item \xmlAttr{method}: \xmlDesc{string, optional}, 
                        -- no description yet --
                  \end{itemize}

                \item \xmlNode{variable}:\xmlDesc{string}, 
                  -- no description yet --

                \item \xmlNode{growth}:\xmlDesc{float}, 
                  -- no description yet --
                  The \xmlNode{growth} node recognizes the following parameters:
                    \begin{itemize}
                      \item \xmlAttr{mode}: \xmlDesc{growthType, optional}, 
                        -- no description yet --
                  \end{itemize}
              \end{itemize}

            \item \xmlNode{reference\_driver}:
              -- no description yet --

              The \xmlNode{reference\_driver} node recognizes the following subnodes:
              \begin{itemize}
                \item \xmlNode{fixed\_value}:\xmlDesc{float}, 
                  -- no description yet --

                \item \xmlNode{sweep\_values}:\xmlDesc{float\_list}, 
                  -- no description yet --

                \item \xmlNode{opt\_bounds}:\xmlDesc{float\_list}, 
                  -- no description yet --

                \item \xmlNode{ARMA}:\xmlDesc{string}, 
                  -- no description yet --
                  The \xmlNode{ARMA} node recognizes the following parameters:
                    \begin{itemize}
                      \item \xmlAttr{variable}: \xmlDesc{string, optional}, 
                        -- no description yet --
                  \end{itemize}

                \item \xmlNode{Function}:\xmlDesc{string}, 
                  -- no description yet --
                  The \xmlNode{Function} node recognizes the following parameters:
                    \begin{itemize}
                      \item \xmlAttr{method}: \xmlDesc{string, optional}, 
                        -- no description yet --
                  \end{itemize}

                \item \xmlNode{variable}:\xmlDesc{string}, 
                  -- no description yet --

                \item \xmlNode{growth}:\xmlDesc{float}, 
                  -- no description yet --
                  The \xmlNode{growth} node recognizes the following parameters:
                    \begin{itemize}
                      \item \xmlAttr{mode}: \xmlDesc{growthType, optional}, 
                        -- no description yet --
                  \end{itemize}
              \end{itemize}

            \item \xmlNode{scaling\_factor\_x}:
              -- no description yet --

              The \xmlNode{scaling\_factor\_x} node recognizes the following subnodes:
              \begin{itemize}
                \item \xmlNode{fixed\_value}:\xmlDesc{float}, 
                  -- no description yet --

                \item \xmlNode{sweep\_values}:\xmlDesc{float\_list}, 
                  -- no description yet --

                \item \xmlNode{opt\_bounds}:\xmlDesc{float\_list}, 
                  -- no description yet --

                \item \xmlNode{ARMA}:\xmlDesc{string}, 
                  -- no description yet --
                  The \xmlNode{ARMA} node recognizes the following parameters:
                    \begin{itemize}
                      \item \xmlAttr{variable}: \xmlDesc{string, optional}, 
                        -- no description yet --
                  \end{itemize}

                \item \xmlNode{Function}:\xmlDesc{string}, 
                  -- no description yet --
                  The \xmlNode{Function} node recognizes the following parameters:
                    \begin{itemize}
                      \item \xmlAttr{method}: \xmlDesc{string, optional}, 
                        -- no description yet --
                  \end{itemize}

                \item \xmlNode{variable}:\xmlDesc{string}, 
                  -- no description yet --

                \item \xmlNode{growth}:\xmlDesc{float}, 
                  -- no description yet --
                  The \xmlNode{growth} node recognizes the following parameters:
                    \begin{itemize}
                      \item \xmlAttr{mode}: \xmlDesc{growthType, optional}, 
                        -- no description yet --
                  \end{itemize}
              \end{itemize}

            \item \xmlNode{depreciate}:\xmlDesc{integer}, 
              -- no description yet --
          \end{itemize}
      \end{itemize}
  \end{itemize}

\section{Economics Introduction}Lorem ipsum dolor sit amet, consectetur adipiscing elit, sed do eiusmod tempor incididunt ut labore et dolore magna aliqua.



\subsection{CashFlow}
  -- no description yet --

  The \xmlNode{CashFlow} node recognizes the following parameters:
    \begin{itemize}
      \item \xmlAttr{name}: \xmlDesc{string, required}, 
        -- no description yet --
      \item \xmlAttr{type}: \xmlDesc{string, required}, 
        -- no description yet --
      \item \xmlAttr{taxable}: \xmlDesc{bool, required}, 
        -- no description yet --
      \item \xmlAttr{inflation}: \xmlDesc{string, required}, 
        -- no description yet --
      \item \xmlAttr{mult\_target}: \xmlDesc{bool, required}, 
        -- no description yet --
      \item \xmlAttr{period}: \xmlDesc{period\_opts, optional}, 
        -- no description yet --
  \end{itemize}

  The \xmlNode{CashFlow} node recognizes the following subnodes:
  \begin{itemize}
    \item \xmlNode{driver}:
      -- no description yet --

      The \xmlNode{driver} node recognizes the following subnodes:
      \begin{itemize}
        \item \xmlNode{fixed\_value}:\xmlDesc{float}, 
          -- no description yet --

        \item \xmlNode{sweep\_values}:\xmlDesc{float\_list}, 
          -- no description yet --

        \item \xmlNode{opt\_bounds}:\xmlDesc{float\_list}, 
          -- no description yet --

        \item \xmlNode{ARMA}:\xmlDesc{string}, 
          -- no description yet --
          The \xmlNode{ARMA} node recognizes the following parameters:
            \begin{itemize}
              \item \xmlAttr{variable}: \xmlDesc{string, optional}, 
                -- no description yet --
          \end{itemize}

        \item \xmlNode{Function}:\xmlDesc{string}, 
          -- no description yet --
          The \xmlNode{Function} node recognizes the following parameters:
            \begin{itemize}
              \item \xmlAttr{method}: \xmlDesc{string, optional}, 
                -- no description yet --
          \end{itemize}

        \item \xmlNode{variable}:\xmlDesc{string}, 
          -- no description yet --

        \item \xmlNode{growth}:\xmlDesc{float}, 
          -- no description yet --
          The \xmlNode{growth} node recognizes the following parameters:
            \begin{itemize}
              \item \xmlAttr{mode}: \xmlDesc{growthType, optional}, 
                -- no description yet --
          \end{itemize}
      \end{itemize}

    \item \xmlNode{reference\_price}:
      -- no description yet --

      The \xmlNode{reference\_price} node recognizes the following subnodes:
      \begin{itemize}
        \item \xmlNode{fixed\_value}:\xmlDesc{float}, 
          -- no description yet --

        \item \xmlNode{sweep\_values}:\xmlDesc{float\_list}, 
          -- no description yet --

        \item \xmlNode{opt\_bounds}:\xmlDesc{float\_list}, 
          -- no description yet --

        \item \xmlNode{ARMA}:\xmlDesc{string}, 
          -- no description yet --
          The \xmlNode{ARMA} node recognizes the following parameters:
            \begin{itemize}
              \item \xmlAttr{variable}: \xmlDesc{string, optional}, 
                -- no description yet --
          \end{itemize}

        \item \xmlNode{Function}:\xmlDesc{string}, 
          -- no description yet --
          The \xmlNode{Function} node recognizes the following parameters:
            \begin{itemize}
              \item \xmlAttr{method}: \xmlDesc{string, optional}, 
                -- no description yet --
          \end{itemize}

        \item \xmlNode{variable}:\xmlDesc{string}, 
          -- no description yet --

        \item \xmlNode{growth}:\xmlDesc{float}, 
          -- no description yet --
          The \xmlNode{growth} node recognizes the following parameters:
            \begin{itemize}
              \item \xmlAttr{mode}: \xmlDesc{growthType, optional}, 
                -- no description yet --
          \end{itemize}
      \end{itemize}

    \item \xmlNode{reference\_driver}:
      -- no description yet --

      The \xmlNode{reference\_driver} node recognizes the following subnodes:
      \begin{itemize}
        \item \xmlNode{fixed\_value}:\xmlDesc{float}, 
          -- no description yet --

        \item \xmlNode{sweep\_values}:\xmlDesc{float\_list}, 
          -- no description yet --

        \item \xmlNode{opt\_bounds}:\xmlDesc{float\_list}, 
          -- no description yet --

        \item \xmlNode{ARMA}:\xmlDesc{string}, 
          -- no description yet --
          The \xmlNode{ARMA} node recognizes the following parameters:
            \begin{itemize}
              \item \xmlAttr{variable}: \xmlDesc{string, optional}, 
                -- no description yet --
          \end{itemize}

        \item \xmlNode{Function}:\xmlDesc{string}, 
          -- no description yet --
          The \xmlNode{Function} node recognizes the following parameters:
            \begin{itemize}
              \item \xmlAttr{method}: \xmlDesc{string, optional}, 
                -- no description yet --
          \end{itemize}

        \item \xmlNode{variable}:\xmlDesc{string}, 
          -- no description yet --

        \item \xmlNode{growth}:\xmlDesc{float}, 
          -- no description yet --
          The \xmlNode{growth} node recognizes the following parameters:
            \begin{itemize}
              \item \xmlAttr{mode}: \xmlDesc{growthType, optional}, 
                -- no description yet --
          \end{itemize}
      \end{itemize}

    \item \xmlNode{scaling\_factor\_x}:
      -- no description yet --

      The \xmlNode{scaling\_factor\_x} node recognizes the following subnodes:
      \begin{itemize}
        \item \xmlNode{fixed\_value}:\xmlDesc{float}, 
          -- no description yet --

        \item \xmlNode{sweep\_values}:\xmlDesc{float\_list}, 
          -- no description yet --

        \item \xmlNode{opt\_bounds}:\xmlDesc{float\_list}, 
          -- no description yet --

        \item \xmlNode{ARMA}:\xmlDesc{string}, 
          -- no description yet --
          The \xmlNode{ARMA} node recognizes the following parameters:
            \begin{itemize}
              \item \xmlAttr{variable}: \xmlDesc{string, optional}, 
                -- no description yet --
          \end{itemize}

        \item \xmlNode{Function}:\xmlDesc{string}, 
          -- no description yet --
          The \xmlNode{Function} node recognizes the following parameters:
            \begin{itemize}
              \item \xmlAttr{method}: \xmlDesc{string, optional}, 
                -- no description yet --
          \end{itemize}

        \item \xmlNode{variable}:\xmlDesc{string}, 
          -- no description yet --

        \item \xmlNode{growth}:\xmlDesc{float}, 
          -- no description yet --
          The \xmlNode{growth} node recognizes the following parameters:
            \begin{itemize}
              \item \xmlAttr{mode}: \xmlDesc{growthType, optional}, 
                -- no description yet --
          \end{itemize}
      \end{itemize}

    \item \xmlNode{depreciate}:\xmlDesc{integer}, 
      -- no description yet --
  \end{itemize}

\clearpage
    \providecommand*{\phantomsection}{}
    \phantomsection
    \addcontentsline{toc}{section}{References}
    \bibliographystyle{ieeetr}
    \bibliography{raven_user_manual}
    \end{document}